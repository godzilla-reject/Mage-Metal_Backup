% Options for packages loaded elsewhere
\PassOptionsToPackage{unicode}{hyperref}
\PassOptionsToPackage{hyphens}{url}
%
\documentclass[
]{article}
\usepackage{amsmath,amssymb}
\usepackage{lmodern}
\usepackage{iftex}
\ifPDFTeX
  \usepackage[T1]{fontenc}
  \usepackage[utf8]{inputenc}
  \usepackage{textcomp} % provide euro and other symbols
\else % if luatex or xetex
  \usepackage{unicode-math}
  \defaultfontfeatures{Scale=MatchLowercase}
  \defaultfontfeatures[\rmfamily]{Ligatures=TeX,Scale=1}
\fi
% Use upquote if available, for straight quotes in verbatim environments
\IfFileExists{upquote.sty}{\usepackage{upquote}}{}
\IfFileExists{microtype.sty}{% use microtype if available
  \usepackage[]{microtype}
  \UseMicrotypeSet[protrusion]{basicmath} % disable protrusion for tt fonts
}{}
\makeatletter
\@ifundefined{KOMAClassName}{% if non-KOMA class
  \IfFileExists{parskip.sty}{%
    \usepackage{parskip}
  }{% else
    \setlength{\parindent}{0pt}
    \setlength{\parskip}{6pt plus 2pt minus 1pt}}
}{% if KOMA class
  \KOMAoptions{parskip=half}}
\makeatother
\usepackage{xcolor}
\setlength{\emergencystretch}{3em} % prevent overfull lines
\providecommand{\tightlist}{%
  \setlength{\itemsep}{0pt}\setlength{\parskip}{0pt}}
\setcounter{secnumdepth}{-\maxdimen} % remove section numbering
\ifLuaTeX
  \usepackage{selnolig}  % disable illegal ligatures
\fi
\IfFileExists{bookmark.sty}{\usepackage{bookmark}}{\usepackage{hyperref}}
\IfFileExists{xurl.sty}{\usepackage{xurl}}{} % add URL line breaks if available
\urlstyle{same} % disable monospaced font for URLs
\hypersetup{
  pdftitle={Character Creation Playtest 1 Packet},
  hidelinks,
  pdfcreator={LaTeX via pandoc}}

\title{Character Creation Playtest 1 Packet}
\author{}
\date{}

\begin{document}
\maketitle

\emph{For my best friends, Brody, Josh, Caleb, Konnar, Matthew and
everyone else. I made this game, and all in the future, for us.}

\emph{Table of contents: no headings found}

\hypertarget{foreword}{%
\section{\texorpdfstring{\emph{Foreword}}{Foreword}}\label{foreword}}

It is rare that I am without word.

But, the support that I have gotten from the people who have eagerly
awaited this mere playtest. \emph{That} leaves me without words.

Thank you. If you have \emph{ever} let me rant about the Terrarium, been
forcefully shown some random game element with ``do you think is too
complicated?!?'' in behind it or suffered through \textbf{\emph{yet
another}} ``Game Design is Still Hard'' status in Gen chat, this thank
you is for you. Thank you for never angering with me while I forgo a fun
night of games to toil away at this passion project of mine.

From the bottom of my heart, \textbf{thank you}.

It is with a full and happy heart I welcome you to {{\emph{The Mage in
the Metal}}}.

\hypertarget{overview}{%
\section{Overview}\label{overview}}

\emph{''Greetings Pilot and welcome back to the Terrarium. Time since
last Lunar Incursion: 709 Solar Days.''}

\begin{center}\rule{0.5\linewidth}{0.5pt}\end{center}

Good morning, Pilot. You've graciously agreed to help test the latest in
USH automatic biographing technology in the form of the \emph{Advanced
Magisterial Pilot Condition and History Bio-Log} (AMP-CHB). The
following document contains all of the instruction you should need to
complete the test here today.

Welcome to {{\emph{The Mage in the Metal}}} \emph{Character Creation
Playtest}. The goal of this test is to check the viability, stability
and usability of the character creation rules for this game. The
following materials are required (and should have been provided) for
this playtest:

\begin{itemize}
\tightlist
\item
  A copy of this document.
\item
  A personal copy of
  \href{https://docs.google.com/spreadsheets/d/11sFDvnzlPyw0b4NcCGj40xoyfQpYLrE9l6w3kMeaYho/edit?usp=sharing}{this}
  Google Sheet. This is the playtest character sheet, and it is
  \emph{not} final. It is handy due to taking the stress of math off of
  playtesters.
\item
  At least 3 six-sided polyhedral dice (``d6's''.)
\item
  A device capable of allowing you to fill out
  \href{https://forms.gle/osrYS3qtCQ6ktG6m8}{this} feedback form.
\end{itemize}

This is the \emph{first} playtest of the \emph{first} game I have ever
designed. I am not promising something amazing, but it is \emph{your}
feedback that will help it \emph{become} something amazing. This has
been a passion project of mine for the better part of the last year of
my life, and I sincerely hope that this first playtest can spark an
interest within you, dear player, to bear with an aspiring game designer
as we all delve into the \emph{Terrarium} over the remnants of Earth
Prime over this series of playtests.

The following is an overview for how to create the \emph{Mage} that ends
up being \emph{in} the \emph{Metal.} As you follow these instructions,
read the content set out before you, and create one of the first
characters within a budding universe, keep a few things in mind:

\begin{itemize}
\tightlist
\item
  \emph{What can be better?}
\item
  \emph{What do you enjoy?}
\item
  \emph{What doesn't make sense mechanically?}
\item
  \emph{What DOES make sense mechanically?}
\end{itemize}

Again, I thank you dearly for being apart of this first playtest. Have
fun, Pilots, and remember:

\textbf{\emph{UNITED, WE WILL SLAY THE FOUR HORSEMEN}}

\hypertarget{character-creation-part-1-the-mage}{%
\section{\texorpdfstring{Character Creation Part 1: \emph{The
Mage}}{Character Creation Part 1: The Mage}}\label{character-creation-part-1-the-mage}}

\hypertarget{playing-the-game}{%
\subsection{\texorpdfstring{1. \emph{Playing The
Game}}{1. Playing The Game}}\label{playing-the-game}}

Before you begin determining the mechanics of your character, it is
important to have a grasp on the narrative of said character. This idea,
referred to by us ``professionals'' as a \emph{narrative concept}, will
help direct and inform the mechanical choices you make during character
creation. Maybe your character is some hot-shot piloting prodigy, or a
crotchety old veteran that chose to join the Pilot Program after being
relieved of duty.

Make sure your concept is appropriate for the game your GM wants to run,
as well as the party you'll be playing with. No one wants to see that
their friend has sat down at the table with the intention of playing
someone evil in a game all about saving orphans. Probably. I don't know
your friends.

Once you have your concept in order, it might also be beneficial to
familiarize yourself with the core rules and mechanics in play under the
game, otherwise known as the \emph{system.}

\hypertarget{the-favor-system}{%
\subsubsection{The Favor System}\label{the-favor-system}}

The Favor System is the name of the "engine" so-to-speak that
{{\emph{Mage//Metal}}} runs off of. Most games' system would be as
simple as "d20," but some are "d20, roll under" or a "dice-pool" system.
The Favor System is the modular, under-the-hood, workings of the TTRPG
that can be adapted into other games (with enough work). Not all of the
mechanics of this game are part \emph{of} the Favor System; the
mechanics that are not are wholly unique to this game, and, in theory,
would not appear in their same form in other Favor System games. The
core components of the Favor System are the dice mechanics, action
structure and player mechanic layouts. These various components are
briefly summarized below.

\hypertarget{dice-structure}{%
\paragraph{Dice Structure}\label{dice-structure}}

The Favor System uses 3d6 for nearly all rolls, with the target number
being static and the die result needing to be \textbf{lower} than the
target number. This set of 3d6 can be changed by the game's variance
system (think ``advantage'' and ``disadvantage'' in \emph{Fifth
Edition}.) Rolling \emph{with favor} (or with ``advantage'') lets the
player roll 2d6 instead, and rolling \emph{without favor} (or with
``disadvantage'') lets the player roll 4d6 instead. This variance
mechanic is, in my opinion, the most unique aspect of the system, and
this is the reason the system is named for it. The dice mechanics are
explored further in Appendix A: Dice Overview.

\hypertarget{action-structure}{%
\paragraph{Action Structure}\label{action-structure}}

The Favor System uses a simple set of mechanics for governing
\emph{timed action}, which is any scene that constitutes the players
having limited time to perform actions, such as combat. During sequences
of timed action, a player can move up to their Speed (or equivalent
statistic), perform two Actions, and take one Reaction. A typical Action
might include something like firing a Weapon, while a Reaction might be
casting a spell to protect an ally from damage. More details can be
found in Appendix B: Action Overview.

\hypertarget{player-mechanic-layout}{%
\paragraph{Player Mechanic Layout}\label{player-mechanic-layout}}

Part of the Favor System is how it defines the relationship between a
mechanic and its associated power progression. This idea is heavily
inspired by usage-based progression from video games such as
\emph{Morrowind} and \emph{Oblivion}, in which the player unlocks new
powers (or "perks") for an ability or skill via usage. The Favor System
changes this slightly to be based on \emph{training} instead, in order
to prevent the players from having to book-keep how often they use
skills or abilities. A practical explanation of this, for example, would
be the Training-Perk progression of a player's School of Arcana: when
Player A gains a new training level in his School, he gains a numerical
bonus of some form; perhaps he gains a +2 bonus to damage with spells
from that school, and then he gains the perk associated with that
training level. This is to avoid players needing to develop things like
a skill and \textbf{\emph{then}} get that cool perk they wanted two
levels ago but did not meet the requirements for. Some mechanics within
a Favor System game might forgo this direct relationship for things that
have no real progression to them, such as a background or player
species.

Now that you know more about the inner workings of this game, lets get
you familiarized with the nitty gritty: Attributes and Statistics; these
are the mathematical core of your Mage.

\hypertarget{mage-mechanics-overview}{%
\subsubsection{Mage Mechanics Overview}\label{mage-mechanics-overview}}

Taking the typical ``number set plus derived number set'' approach of
most RPGs, {{\emph{Mage//Metal}}} divides these two sets into Attributes
(the primary number set) and Statistics (the derived number set.)

\emph{A Note On Rules Language}

{{\emph{Mage//Metal}}} uses a combination of \emph{italics},
\textbf{bold} and Proper capitalization in its rules text to help
differentiate and call attention to important pieces of rules text. An
example of this is `Attributes' and `Statistics' always being
capitalized properly due to being mechanics.

\hypertarget{attributes}{%
\paragraph{Attributes}\label{attributes}}

These are the most fundamental parts of a character. Each Attribute
begins at 10, and are \textbf{rarely} changed after character creation.

\hypertarget{power}{%
\subparagraph{Power}\label{power}}

Power is a measure of physicality and force.\\
\emph{Governs}

\begin{itemize}
\tightlist
\item
  Carry Weight
\item
  Melee
\end{itemize}

\hypertarget{body}{%
\subparagraph{Body}\label{body}}

Body is a measure of the robustness and physical fortitude of a
character.\\
\emph{Governs}

\begin{itemize}
\tightlist
\item
  Toughness
\item
  Health
\end{itemize}

\hypertarget{reflex}{%
\subparagraph{Reflex}\label{reflex}}

Reflex is a measure of agility and manual dexterity.\\
\emph{Governs}

\begin{itemize}
\tightlist
\item
  Accuracy
\item
  Speed
\end{itemize}

\hypertarget{mind}{%
\subparagraph{Mind}\label{mind}}

Mind is a measure mental fortitude, education, and memory. It
is~\emph{not}~intelligence. A smart character can have poor memory, and
thus, poor Mind. Mind can also influence magic.\\
\emph{Governs:}

\begin{itemize}
\tightlist
\item
  Skill Aptitude
\item
  Spell Memory
\end{itemize}

\hypertarget{potential}{%
\subparagraph{Potential}\label{potential}}

Potential is a measure of a creature's ability to exert their influence
on the world via metaphysical or meta-mental means, AKA magic. While
hand-in-hand with Mind, Potential is seen by some as a form of
spirituality as well.\\
\emph{Governs}

\begin{itemize}
\tightlist
\item
  Spell Memory
\item
  Mana
\end{itemize}

\hypertarget{tenacity}{%
\subparagraph{Tenacity}\label{tenacity}}

Tenacity is a measure of how powerful a creature's personality and
charisma is. Tenacious characters have a much easier time convincing
others of things, as well as inspiring allies. Some also consider
tenacity to be a meta-physical, or magical, measure of luck and fate.\\
\emph{Governs}

\begin{itemize}
\tightlist
\item
  Luck
\end{itemize}

\hypertarget{statistics}{%
\paragraph{Statistics}\label{statistics}}

Derived from their respective Attribute, different Statistics have
differing formulae for calculating them. If a Statistic would have a
decimal, round \emph{down.}

\hypertarget{power-1}{%
\subparagraph{Power}\label{power-1}}

\begin{itemize}
\tightlist
\item
  Carry Weight

  \begin{itemize}
  \tightlist
  \item
    Measures how much a character can carry and remain unaffected. A
    character is considered encumbered when they're over half of their
    Carry Weight.
  \item
    Max: Power * 12.
  \end{itemize}
\item
  Melee Damage Bonus

  \begin{itemize}
  \tightlist
  \item
    A bonus to melee weapon damage.
  \item
    Power - 10.
  \end{itemize}
\end{itemize}

\hypertarget{body-1}{%
\subparagraph{Body}\label{body-1}}

\begin{itemize}
\tightlist
\item
  Toughness

  \begin{itemize}
  \tightlist
  \item
    A character's ability to resist physical damage.
  \item
    Body/6
  \end{itemize}
\item
  Health

  \begin{itemize}
  \tightlist
  \item
    This is a measure of how much damage a character's body can sustain
    before failure.
  \item
    Max HP: Body.
  \end{itemize}
\end{itemize}

\hypertarget{reflex-1}{%
\subparagraph{Reflex}\label{reflex-1}}

\begin{itemize}
\tightlist
\item
  Ranged Accuracy Bonus

  \begin{itemize}
  \tightlist
  \item
    A bonus to ranged attack rolls.
  \item
    Reflex - 8.
  \end{itemize}
\item
  Speed

  \begin{itemize}
  \tightlist
  \item
    How far one can move in 6 seconds.
  \item
    (Reflex * 3)/10
  \end{itemize}
\end{itemize}

\hypertarget{mind-1}{%
\subparagraph{Mind}\label{mind-1}}

\begin{itemize}
\tightlist
\item
  Skill Aptitude

  \begin{itemize}
  \tightlist
  \item
    A measure of how many skills one can use to their fullest extent. A
    skill that a character has \emph{aptitude} with gains access to the
    Skill Perks of that skill. Not having aptitude does not affect their
    capacity to use that skill in any other way.
  \item
    Mind/4
  \end{itemize}
\end{itemize}

\hypertarget{potential-1}{%
\subparagraph{Potential}\label{potential-1}}

\begin{itemize}
\tightlist
\item
  Spell Memory

  \begin{itemize}
  \tightlist
  \item
    How many spells can be prepared a day.
  \item
    Mind+Potential/4.
  \end{itemize}
\item
  Mana

  \begin{itemize}
  \tightlist
  \item
    Resource used to cast spells.
  \item
    Potential * 2.
  \end{itemize}
\end{itemize}

\hypertarget{tenacity-1}{%
\subparagraph{Tenacity}\label{tenacity-1}}

\begin{itemize}
\tightlist
\item
  Luck

  \begin{itemize}
  \tightlist
  \item
    Luck is a meta-currency that players can use to fuel actions with
    the Fated trait.
  \item
    Tenacity/5 (Minimum of 2.)
  \end{itemize}
\end{itemize}

\hypertarget{proficiencies}{%
\paragraph{Proficiencies}\label{proficiencies}}

\emph{Skills and Aptitude}

In {{\emph{Mage//Metal}}}, skills are essential to all characters. They
can be found in Appendix D: Skill List. More information is found in the
\emph{Finalizing Skills} section of character creation.

{*Insert Proficiencies when Done*}

\hypertarget{species}{%
\subsection{\texorpdfstring{2.
\emph{Species}}{2. Species}}\label{species}}

Once you feel that you are familiar enough with the game's mechanics,
you are ready to move on to creation proper. The first step in this
process is answering this question:

\emph{Who are you?}

Pick one of the following three species. While there are only three
options in this playtest, launch is planned to include at least nine
species across three factions.

Once you've picked your species, make sure you select a Species perk and
an Attribute boost, as well as denote other bonuses from your species.

Take this time to go ahead and fill out non-mechanical details, such as
name, age, pronouns, faith, etc.

\hypertarget{species-of-the-united-solar-hegemony}{%
\subsubsection{\texorpdfstring{\textbf{\emph{Species of the United Solar
Hegemony}}}{Species of the United Solar Hegemony}}\label{species-of-the-united-solar-hegemony}}

\emph{Formed after the destruction of Earth Prime, the U.S.H. stands as
a shining example of comradery and unity.}

\hypertarget{dwaurves-of-mars}{%
\paragraph{\texorpdfstring{\emph{Dwaurves of
Mars}}{Dwaurves of Mars}}\label{dwaurves-of-mars}}

{Dwaurves}

\hypertarget{elves-of-venus}{%
\paragraph{\texorpdfstring{\emph{Elves of
Venus}}{Elves of Venus}}\label{elves-of-venus}}

{Elves}

\hypertarget{humans-of-nowhere}{%
\paragraph{\texorpdfstring{\emph{Humans of
Nowhere}}{Humans of Nowhere}}\label{humans-of-nowhere}}

{Human}

\hypertarget{history}{%
\subsection{\texorpdfstring{3.
\emph{History}}{3. History}}\label{history}}

After determining your character's Species, you must answer another
question:

\emph{Where did you come from?}

The answer to this question is your character's \emph{History.} This is
what the character did \emph{before} becoming a Pilot, and the skills
they learned there are invaluable, even now. Their History also
determines their starting assets and currency via a measure of Wealth.

Pick one of the following Histories. They are presented in order of Poor
to Extreme.

After your selection, ensure you denote your History's perk and
Attribute boost. Once this is done, select or denote the skills gained
from your History. You \emph{do not} gain aptitude in them by default.

Make sure you also list any other starting bonuses or equipment from
your chosen History.

Finally, ensure you denote your character's \emph{starting wealth}, as
stated on the following chart.\\
{Starting Wealth Chart}

\hypertarget{poor-histories}{%
\subsubsection{Poor Histories}\label{poor-histories}}

{Dusted}\\
{Hegemony Nomad}

\hypertarget{moderate-histories}{%
\subsubsection{Moderate Histories}\label{moderate-histories}}

{Rapscallion}\\
{Wrench Monkey}

\hypertarget{high-histories}{%
\subsubsection{High Histories}\label{high-histories}}

{Corpo Deskie}\\
{Influencer}

\hypertarget{extreme-histories}{%
\subsubsection{Extreme Histories}\label{extreme-histories}}

{Politologist}\\
{Noble}

\hypertarget{school-of-arcana}{%
\subsection{\texorpdfstring{4. \emph{School of
Arcana}}{4. School of Arcana}}\label{school-of-arcana}}

Now that you have decided what your character did before becoming a Mage
Pilot, you are faced with yet another question.

\emph{How do you wield the arcane?}

The answer to this question is the methodology of magic your pilot chose
to specialize in when they were training to become a Mage-Pilot. Divided
into Schools, these orders of magic represent deep philosophical ideas
about reality and magic, as well as their preferred method for blowing
up bad guys.

Pick from one of the following School of Arcana, then select one of the
two available Paths offered by your chosen School. You become Trained in
your School, and gain the first Perk of your chosen Path.

After your selections, ensure you denote your School's Attribute Boost
and any starting gear. Take this time to inscribe your starting spells
in your Grimoire as well.\\
{School Note}

\hypertarget{pilot-role}{%
\subsection{\texorpdfstring{5. \emph{Pilot
Role}}{5. Pilot Role}}\label{pilot-role}}

After making a selection as to the style of magic you wield, you are now
posed with a final question:

\emph{What role do you fill within a squad?}

The answer to this question is your \emph{Pilot Role}. This is simply
what \emph{you do} on the battlefield, in and out of the Metal. The
classical idea of the ``tank'', ``support'' and ``DPS'' are all
determined by this choice.

Select a Pilot Role from the following options, then select one of the
two Role Specializations offered by the chosen Role. You then become
Trained in your Role.

Denote any proficiencies granted by your Role, and then denote the
skills granted by your Role and Specialization. You \emph{do not} gain
aptitude in them. Finally, inscribe your Specialization's starting Spell
in your Grimoire.\\
{Role Note}

\hypertarget{finalizing-attributes-and-statistics}{%
\subsection{\texorpdfstring{6. \emph{Finalizing Attributes and
Statistics}}{6. Finalizing Attributes and Statistics}}\label{finalizing-attributes-and-statistics}}

Now that your four core selections have been made, its time to finalize
your Attributes. Assign 12 free Attribute points across your Attributes.
You cannot begin play with an Attribute above 20.

Once you have assigned all 12 points, take this time to calculate your
Statistics (if you aren't using the \emph{free digital character sheet}.
Which, you should be. I worked really hard on it.)

\hypertarget{finalizing-skills-and-proficiencies}{%
\subsection{\texorpdfstring{7. \emph{Finalizing Skills and
Proficiencies}}{7. Finalizing Skills and Proficiencies}}\label{finalizing-skills-and-proficiencies}}

After your Attributes and Statistics are finalized, you are ready to
finalize your Skills and Proficiencies.

\hypertarget{skills}{%
\subsubsection{\texorpdfstring{\textbf{Skills}}{Skills}}\label{skills}}

Designate a number of Skill Aptitudes as determined by your Skill
Aptitudes Statistic, then select one Skill you are Trained in and have
Aptitude with. You begin play as an Expert in that Skill.

\hypertarget{weapon-and-armor-proficiencies}{%
\subsubsection{\texorpdfstring{\textbf{Weapon and Armor
Proficiencies}}{Weapon and Armor Proficiencies}}\label{weapon-and-armor-proficiencies}}

Select one Armor proficiency and one Weapon proficiency to increase your
training level with by one step (untrained to Trained or Trained to
Expert.) You cannot begin play with an Armor or Weapon proficiency at or
above Mastered.

\hypertarget{general-proficiencies}{%
\subsubsection{\texorpdfstring{\textbf{General
Proficiencies}}{General Proficiencies}}\label{general-proficiencies}}

From among \textbf{Perception}, \textbf{Piloting} and
\textbf{Spell-Work}, pick two to begin play at Trained and one to begin
play at Expert.

\hypertarget{creating-a-grimoire}{%
\subsection{\texorpdfstring{8. \emph{Creating a
Grimoire}}{8. Creating a Grimoire}}\label{creating-a-grimoire}}

After all of your mechanical selections, you can move on to creating
your \emph{Grimoire}, which is a catalog of all the spells your Mage can
cast. These are the guidelines for creating your Grimoire:

\begin{enumerate}
\tightlist
\item
  At Tier 1, your Grimoire can contain Rank 0-3 spells.
\item
  Your Grimoire can contain a number of spells equal to your Tier * 10
  (which is 10.)
\item
  You have access to any spell on a spell list that either does not have
  an access requirement or you meet the requirement for. Spells can be
  found in Appendix C Spell Lists
\end{enumerate}

\hypertarget{selecting-perks}{%
\subsection{\texorpdfstring{9. \emph{Selecting
Perks}}{9. Selecting Perks}}\label{selecting-perks}}

Each character in {{\emph{Mage//Metal}}} has (a) quirk(s) that helps set
them apart from other characters. These quirks are known as
\emph{General Perks}. Select one for your Mage Pilot.\\
{Perk Note}

\hypertarget{selecting-equipment}{%
\subsection{\texorpdfstring{10. \emph{Selecting
Equipment}}{10. Selecting Equipment}}\label{selecting-equipment}}

After selecting a General Perk, the final step of Mage creation is to
determine what equipment your Pilot uses.

Using the starting wealth granted by your \textbf{History}, purchase as
little or as much equipment from Appendix E: Equipment. USH-M basic gear
recommendation is at least one weapon (preferably that you have a
proficiency in), one suit of armor (also, preferably that you have
proficiency in), and some basic gear such as Nutri-gel, a Hardlight
Torch, a backpack and a set of clothes.

After gear acquisition, denote any remaining credits.

\hypertarget{finishing-up}{%
\subsection{\texorpdfstring{11. \emph{Finishing
Up}}{11. Finishing Up}}\label{finishing-up}}

Congratulations, you've completed the first (and more complicated) half
of character creation for this play test. Take this time to go ahead and
fill out the \emph{Mage} section of the linked feedback questionnaire.

Now, it is time to create the \emph{Metal} for the \emph{Mage}. Proceed
to Part 2: \emph{The Metal}.

\hypertarget{character-creation-part-2-the-metal}{%
\section{\texorpdfstring{Character Creation Part 2: \emph{The
Metal}}{Character Creation Part 2: The Metal}}\label{character-creation-part-2-the-metal}}

Just like your Mage, a concept of your Metal is needed before you begin
creating it. A couple things to keep in mind when making up the concept
for your Mage's first Metal:

\begin{enumerate}
\tightlist
\item
  Make sure your Metal fits any criteria your playgroup agrees on
  keeping consistent across everyone's Metals. This might be as simple
  as a shared naming method, or as important as everyone using the same
  Frame.
\item
  Ensure your Metal is a reflection and extension of your Pilot. This is
  exactly what Metals are; extensions of the inner parts of a Mage and
  their magic, given raw and powerful form.
\end{enumerate}

Once you've nailed down the concept of your Metal, review how Metal
Mechanics differ from Mage.

\hypertarget{metal-mechanics}{%
\subsubsection{Metal Mechanics}\label{metal-mechanics}}

{Metal Attributes and Statistics}

\hypertarget{frame}{%
\subsection{\texorpdfstring{1. \emph{Frame}}{1. Frame}}\label{frame}}

The \emph{Frame} is the most basic component of the Metal, forming the
foundation of which all other parts are mounted upon. It determines
fundamental aspects of a Metal, such as Size, body-plan, and weapon
mounts.

Select a \textbf{Frame} from the options below, then denote the
Attribute bonuses, Weapon Mounts, and any Perks or Integrated Systems
granted by the Frame.

{BF-001}\\
{BF-002}\\
{HF-001}\\
{HF-002}\\
{LF-001}\\
{LF-002}\\
{SHF-001}\\
{SHF-002}\\
{ULF-001}\\
{ULF-002}

\hypertarget{shell}{%
\subsection{\texorpdfstring{2. \emph{Shell}}{2. Shell}}\label{shell}}

The \emph{Shell} of the Metal is what determines the armor plating of
your Metal, as well as an \emph{Intrinsic} ability that provides
powerful combat abilities.

Pick a \textbf{Shell} from the options below. Selection is restricted by
your chosen \textbf{Pilot Role} during Mage creation.

After you've chosen your \textbf{Shell}, denote the following: Attribute
boost, armor type, Weapon Mounts, and Intrinsic.\\
{Bunker Buster}\\
{Fortress Breaker}\\
{Subtle Dialog}\\
{Surgeon General}\\
{Wire Runner}

\hypertarget{soul-core}{%
\subsection{\texorpdfstring{3. \emph{Soul
Core}}{3. Soul Core}}\label{soul-core}}

The \emph{Soul Core} of the Metal is exactly that: a core, with a soul
therewithin. It acts as the heart of the Metal's \emph{Soul-Resonance
Reactor}, or S-R\textsuperscript{2} Engine. The soul of spirit within
the core reacts with the latent Potential of the Pilot, providing a
nearly infinite amount of energy to run the massive machine.

The \textbf{Soul Core} of a Metal is a special choice for a Pilot, as it
dictates a special confrontation that the Pilot must eventually face at
their darkest hour; the \emph{Deity within the Machine}, the
\textbf{Deus Ex Machina}. Pay special attention to the \textbf{Deus} of
your chosen \textbf{Soul Core}.

Pick a \textbf{Soul Core} for your Metal, and then denote granted the
granted Attribute bonuses, Core Passive, and Core Intrinsic. You can
also denote your \textbf{Deus Ex Machina}, but it will not be relevant
in this playtest (or the next one, honestly.) In future playtests,
\textbf{Soul Cores} will be limited by your chosen \textbf{School of
Arcana} during Mage creation.

{ATSC-001 ''Warring Dragons''}\\
{DTSC-001 ''Steelclad Mamba''}\\
{STSC-001 ''Infinite Hydrangea''}

\hypertarget{finalizing-metal-attributes-and-statistics}{%
\subsection{\texorpdfstring{4. \emph{Finalizing Metal Attributes and
Statistics}}{4. Finalizing Metal Attributes and Statistics}}\label{finalizing-metal-attributes-and-statistics}}

After selecting your Metal's basic components, it is time to finalize
its Attributes and Statistics, just as you did for your Pilot.

Assign 6 free Attribute points. You cannot begin play with a Metal
Attribute above 30.

Once you've assigned these points, take this time to calculate your
Metal's Statistics if you aren't using the digital character sheet
(which, again, you should be. I worked \emph{really} hard on it.)

\hypertarget{installing-metal-systems}{%
\subsection{\texorpdfstring{5. \emph{Installing Metal
Systems}}{5. Installing Metal Systems}}\label{installing-metal-systems}}

Other abilities, such as \emph{jet boosters} and other typical
mecha-fantasy fair, in {{\emph{Mage//Metal}}} are relegated to
\emph{Systems}, which each Metal can only have a select amount of. This
limit is dictated by the \textbf{System Points} Statistic.

Spend these SP on as many, or as few, systems as you want. You \emph{do
not} have to spend all of them.

In future playtests, Systems will be limited in access by the components
of the Metal.

Metal Systems can be found in Appendix F: Metal Equipment

\hypertarget{mounting-weaponry}{%
\subsection{\texorpdfstring{6. \emph{Mounting
Weaponry}}{6. Mounting Weaponry}}\label{mounting-weaponry}}

As you've created your Metal, you've seen mention of ``weapon mounts.''
These are, in effect, `weapon slots' for your Metal. Each mount has a
``size'': \emph{auxiliary}, \emph{secondary}, \emph{primary},
\emph{heavy} and \emph{superprimary}.

Each mount can house a weapon of its size or \emph{one smaller}.
Additionally, \emph{super-primary} weaponry can be mounted across
\emph{two} mounts, at least one of which must be a \emph{primary/aux} or
\emph{heavy} mount.

Some mounts also have other modifiers, such as ``Twin'' or
``(Restriction)''. Two of the same weapon can be installed in a Twin
mount, and can be fired independently \emph{or} simultaneously. Mounts
with (Restrictions) are limited to only what is permitted by the
Restriction.

Finally, a mount can have a ``/Aux'' modifier, which allows for an extra
auxiliary weapon to be installed alongside the main weapon that fires
when the main weapon is fired.

Each mount on your Metal must have a weapon installed in it.

Metal Weapons can be found in Appendix F: Metal Equipment

\hypertarget{finishing-up-1}{%
\subsection{\texorpdfstring{7. \emph{Finishing
Up}}{7. Finishing Up}}\label{finishing-up-1}}

Your Metal is mechanically complete! Congratulations!

\hypertarget{wrapping-up}{%
\section{Wrapping Up}\label{wrapping-up}}

There is one last step here: style and flair. Take this time to go a bit
more in-depth on what both your Pilot and their Metal look like. Write
up a small description of both your Mage-Pilot and your Metal,
incorporating things chosen during creation of each of them.

From here, go ahead and fill out the second part of the Playtest
Feedback Questionnaire. Once you've done that, ensure you keep a copy of
your character sheet, as you'll be revisiting this same character in
Playtest 2 (and beyond!)

Again, I want to personally thank you for being a participant in the
first ever playtest for my first ever game. I had no clue the amount of
work that would go into just getting here, but the continued interest
from all of the people closest to me is what got me to this point. I
look forward to the next playtest, and I hope you do too.

Remember Pilots,

\textbf{UNITED, WE WILL SLAY THE FOUR HORSEMEN}

\hypertarget{appendices}{%
\section{Appendices}\label{appendices}}

\hypertarget{appendix-a-dice-overview}{%
\subsection{Appendix A: Dice Overview}\label{appendix-a-dice-overview}}

{Dice Overview}

\hypertarget{appendix-b-action-overview}{%
\subsection{Appendix B: Action
Overview}\label{appendix-b-action-overview}}

{Actions Overview}

\hypertarget{appendix-c-spell-lists}{%
\subsection{Appendix C: Spell Lists}\label{appendix-c-spell-lists}}

{Spell List}

\hypertarget{appendix-d-skill-list}{%
\subsection{Appendix D: Skill List}\label{appendix-d-skill-list}}

{Skill List}

\hypertarget{appendix-e-mage-equipment}{%
\subsection{Appendix E: Mage
Equipment}\label{appendix-e-mage-equipment}}

\hypertarget{e.1-general-equipment}{%
\subsubsection{E.1: General Equipment}\label{e.1-general-equipment}}

{Equipment Appendix}

\hypertarget{e.2-weaponry-pr}{%
\subsubsection{E.2: Weaponry Pr}\label{e.2-weaponry-pr}}

\hypertarget{e.3-armor}{%
\subsubsection{E.3: Armor}\label{e.3-armor}}

\hypertarget{appendix-f-metal-equipment}{%
\subsection{Appendix F: Metal
Equipment}\label{appendix-f-metal-equipment}}

\hypertarget{f.1-metal-systems}{%
\subsubsection{F.1: Metal Systems}\label{f.1-metal-systems}}

\hypertarget{f.2-metal-weaponry}{%
\subsubsection{F.2: Metal Weaponry}\label{f.2-metal-weaponry}}

\end{document}
